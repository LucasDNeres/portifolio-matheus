\documentclass[a4paper,12pt]{article}
\usepackage[brazil]{babel}
\usepackage[utf8]{inputenc}
\usepackage[T1]{fontenc}
\usepackage{geometry}
\usepackage{enumitem}
\usepackage{hyperref}
\usepackage{fancyhdr}
\usepackage{titlesec}
\usepackage{helvet}
\renewcommand{\familydefault}{\sfdefault} % Usa fonte sem serifa (Helvetica/Arial)

\geometry{
    top=3cm,
    left=3cm,
    right=2cm,
    bottom=2cm
}

\hypersetup{
    colorlinks=true,
    linkcolor=blue,
    urlcolor=blue
}

\pagestyle{fancy}
\fancyhf{}
\rhead{Documentação do Projeto}
\lhead{Portfólio Matheus A. de Souza}
\rfoot{\thepage}

\titleformat{\section}{\normalfont\Large\bfseries}{\thesection}{1em}{}

\begin{document}

% Capa no formato ABNT
\begin{titlepage}
    \begin{center}
        \large
        \textbf{UNIVERSIDADE POSITIVO}\\[0.5cm]
        CURSO SUPERIOR DE TECNOLOGIA EM\\
        ANÁLISE E DESENVOLVIMENTO DE SISTEMAS

        \vspace{3cm} % Espaço entre instituição e autores

        GABRIEL KEMMERER\\
        LUCAS NERES\\
        LEONARDO VALENTE\\
        ENZO CHEMIN

        \vspace{4cm} % Espaço entre autores e título

        \textbf{DOCUMENTAÇÃO DO PROJETO}\\
        \textbf{PORTFÓLIO DO MATHEUS ANTONIELE DE SOUZA}

        \vfill

        Curitiba – PR\\
        2025
    \end{center}
\end{titlepage}

\tableofcontents
\newpage

\section{Introdução}
Este documento apresenta o andamento do projeto de criação do portfólio digital do Matheus Antoniele de Souza. Estamos na fase inicial, organizando ideias e documentação antes de iniciar o desenvolvimento da aplicação web.

\section{Objetivo do Projeto}
O objetivo é criar um portfólio online que destaque os projetos, experiências e qualificações do Matheus, servindo como vitrine profissional e canal de contato. O portfólio será desenvolvido com foco em usabilidade, design limpo e responsividade.

\section{Objetivo Profissional do Matheus}
Matheus deseja atuar na área de Tecnologia da Informação, com foco em suporte técnico e desenvolvimento de sistemas. Ele busca oportunidades que lhe permitam expandir seus conhecimentos em software, afastando-se gradualmente do setor de hardware, onde já possui sólida experiência.

\section{Experiência Profissional}
\textbf{Bios Update Informática} – Curitiba/PR \\
\textit{Suporte Técnico em TI (2022 - Atual)}

\begin{itemize}
  \item Atendimento e suporte técnico a usuários.
  \item Manutenção de hardware e infraestrutura de TI.
  \item Cabeamento de rede e configuração de equipamentos.
  \item Diagnóstico e solução de problemas técnicos.
\end{itemize}

\section{Formação Acadêmica}
\begin{itemize}
  \item Ensino Médio Completo.
  \item Superior de Tecnologia em Análise e Desenvolvimento de Sistemas – \textbf{Universidade Positivo} (em andamento, turno noturno).
\end{itemize}

\section{Habilidades e Competências}
\begin{itemize}
  \item Manutenção e assistência técnica em TI.
  \item Cabeamento e configuração de redes.
  \item Conhecimentos em hardware e infraestrutura.
  \item Programação nas linguagens C, Python, Java, SQL e HTML.
\end{itemize}

\section{Idiomas}
\begin{itemize}
  \item Português: Nativo.
  \item Inglês: Básico (atualmente em curso de aprimoramento).
\end{itemize}

\section{Informações de Contato}
\begin{itemize}
  \item \textbf{Nome:} Matheus Antoniele de Souza
  \item \textbf{Localidade:} Curitiba, Paraná
  \item \textbf{Telefone / WhatsApp:} (41) 99263-0374
  \item \textbf{E-mail:} \href{mailto:matheus.antoniele@gmail.com}{matheus.antoniele@gmail.com}
  \item \textbf{LinkedIn:} \href{https://www.linkedin.com/in/matheus-antoniel-a756b634b}{Matheus-Antoniele}
  \item \textbf{GitHub:} \href{https://github.com/Antoniele20}{Antoniele20}
\end{itemize}

\section{Equipe}
O projeto está sendo desenvolvido pelos seguintes integrantes:

\begin{itemize}
  \item Gabriel Kemmerer
  \item Lucas Neres
  \item Leonardo Valente
  \item Enzo Chemin
\end{itemize}

\section{Ferramentas Utilizadas}
\begin{itemize}
  \item \textbf{Trello:} Gerenciamento de tarefas e progresso do time.
  \item \textbf{Overleaf:} Criação colaborativa da documentação em LaTeX.
  \item \textbf{Pacote Office:} Word, Excel e PowerPoint para relatórios, cronogramas e apresentações.
  \item \textbf{Canva:} Criação de mockups e materiais visuais.
\end{itemize}

\section{Organização do Trabalho}
No Trello, as tarefas são organizadas em colunas:

\begin{itemize}
  \item Backlog
  \item Em andamento
  \item Em revisão
  \item Concluído
\end{itemize}

Reuniões regulares com o Matheus ajudam a alinhar expectativas, apresentar protótipos e registrar feedbacks para o desenvolvimento do portfólio.

\section{Requisitos Funcionais}
Os seguintes requisitos foram organizados com base na metodologia MoSCoW:

\subsection*{Must (Obrigatórios)}
\begin{itemize}
  \item Exibir nome completo do cliente.
  \item Mostrar resumo profissional e formação acadêmica.
  \item Listar experiências profissionais.
  \item Apresentar o portfólio de projetos.
  \item Garantir que o site seja responsivo.
\end{itemize}

\subsection*{Should (Importantes)}
\begin{itemize}
  \item Exibir foto profissional do cliente.
  \item Mostrar habilidades técnicas e interpessoais.
  \item Inserir links externos (LinkedIn, GitHub).
\end{itemize}

\subsection*{Could (Desejáveis)}
\begin{itemize}
  \item Botão para baixar o currículo em PDF.
  \item Formulário de contato com validação básica.
\end{itemize}

\section{Próximos Passos}
\begin{itemize}
  \item Definir layout e funcionalidades finais.
  \item Iniciar desenvolvimento com HTML, CSS e Java.
  \item Implementar controle de versão via GitHub.
  \item Atualizar documentação conforme o progresso.
\end{itemize}

\section{Conclusão}
A documentação clara e contínua é fundamental para o sucesso do projeto. A combinação de ferramentas tecnológicas e o contato constante com o cliente asseguram o alinhamento das expectativas e a qualidade final da entrega.

\end{document}